%%
% The BIThesis Template for Bachelor Graduation Thesis
%
% 北京理工大学毕业设计(论文)附录 —— 使用 XeLaTeX 编译
%
% Copyright 2020-2022 BITNP
%
% This work may be distributed and/or modified under the
% conditions of the LaTeX Project Public License, either version 1.3
% of this license or (at your option) any later version.
% The latest version of this license is in
%   http://www.latex-project.org/lppl.txt
% and version 1.3 or later is part of all distributions of LaTeX
% version 2005/12/01 or later.
%
% This work has the LPPL maintenance status `maintained'.
%
% The Current Maintainer of this work is Feng Kaiyu.
%
% Compile with: xelatex -> biber -> xelatex -> xelatex

\unnumchapter{附~~~~录}
\renewcommand{\thechapter}{附录}

% 设置附录编号格式
\ctexset{
  section/number = 附录\Alph{section}
}
\numberwithin{figure}{section}
\numberwithin{table}{section}
\numberwithin{lstlisting}{section}
\renewcommand{\thefigure}{附录\Alph{section}-\arabic{figure}}
\renewcommand{\thetable}{附录\Alph{section}-\arabic{table}}
\renewcommand{\thelstlisting}{附录\Alph{section}-\arabic{lstlisting}}
% 如果要为其他项目设置编号格式,参考如下注释行,这两行为算法块修改编号格式
% \numberwithin{algorithm}{section}
% \renewcommand{\thealgorithm}{附录\Alph{section}-\arabic{algorithm}}

附录相关内容…

% 这里示范一下添加多个附录的方法:

\section{\LaTeX 环境的安装}
\LaTeX 环境的安装。

\section{BIThesis 使用说明}
BIThesis 使用说明。

\textcolor{blue}{附录是毕业设计(论文)主体的补充项目,为了体现整篇文章的完整性,写入正文又可能有损于论文的条理性、逻辑性和精炼性,这些材料可以写入附录段,但对于每一篇文章并不是必须的。附录依次用大写正体英文字母 A、B、C……编序号,如附录 A、附录 B。阅后删除此段。}

\textcolor{blue}{附录正文样式与文章正文相同:宋体、小四;行距:22 磅;间距段前段后均为 0 行。阅后删除此段。}
