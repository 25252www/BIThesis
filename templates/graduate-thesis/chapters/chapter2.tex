%%
% The BIThesis Template for Graduate Thesis
%
% Copyright 2020-2023 Yang Yating, BITNP
%
% This work may be distributed and/or modified under the
% conditions of the LaTeX Project Public License, either version 1.3
% of this license or (at your option) any later version.
% The latest version of this license is in
%   http://www.latex-project.org/lppl.txt
% and version 1.3 or later is part of all distributions of LaTeX
% version 2005/12/01 or later.
%
% This work has the LPPL maintenance status `maintained'.
%
% The Current Maintainer of this work is Feng Kaiyu.
%
% Compile with: xelatex -> biber -> xelatex -> xelatex

\chapter{具体研究内容}

具体研究内容是学位论文的主要部分,是研究结果及其依据的具体表述,是研究能力的集中体现,一般应包括第2章、第3章至结论前一章。具体研究内容应该结构合理,层次清楚,重点突出,文字简练、通顺。可包括以下各方面:研究对象、研究方法、仪器设备、材料原料、实验和观测结果、理论推导、计算方法和编程原理、数据资料和经过加工整理的图表、理论分析、形成的论点和导出的结论等。具体研究内容各章后可有一节“本章小结”(必要时)。

\begin{them}[留数定理]
\label{thm:res}
  假设$U$是复平面上的一个单连通开子集,$a_1,\ldots,a_n$是复平面上有限个点,$f$是定义在$U\backslash \{a_1,\ldots,a_n\}$上的全纯函数,
  如果$\gamma$是一条把$a_1,\ldots,a_n$包围起来的可求长曲线,但不经过任何一个$a_k$,并且其起点与终点重合,那么:

  \begin{equation}
    \label{eq:res}
    \ointop_{\gamma}f(z)\,\mathrm{d}z = 2\pi\mathbf{i}\sum^n_{k=1}\mathrm{I}(\gamma,a_k)\mathrm{Res}(f,a_k)
  \end{equation}

  如果$\gamma$是若尔当曲线,那么$\mathrm{I}(\gamma, a_k)=1$,因此:

  \begin{equation}
    \label{eq:resthm}
    \ointop_{\gamma}f(z)\,\mathrm{d}z = 2\pi\mathbf{i}\sum^n_{k=1}\mathrm{Res}(f,a_k)
  \end{equation}

  在这里,$\mathrm{Res}(f, a_k)$表示$f$在点$a_k$的留数,$\mathrm{I}(\gamma,a_k)$表示$\gamma$关于点$a_k$的卷绕数。
  卷绕数是一个整数,它描述了曲线$\gamma$绕过点$a_k$的次数。如果$\gamma$依逆时针方向绕着$a_k$移动,卷绕数就是一个正数,
  如果$\gamma$根本不绕过$a_k$,卷绕数就是零。
\end{them}

\begin{proof}
  首先,由……

  其次,……

  所以,由\autoref{thm:res}可知……
  \qedhere
\end{proof}
  
