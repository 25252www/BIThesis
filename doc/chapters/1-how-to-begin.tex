\section{如何开始}
{\BIThesis} 为各位在北京理工大学就读的本科同学提供了基于北京理工大学计算机学院教务部给出的“北京理工大学计算机学院本科生毕业论文:开题报告”与北京理工大学教务部提供的“北京理工大学本科生毕业设计:论文模板(目前是 2019 届版本)”的 \LaTeX 样版。借助于 {\BIThesis} 的 \LaTeX 模板,你可以在保证论文格式整齐、完美、符合要求的前提下,专注于学术研究、项目实现,从而顺利完成你的学术项目。

本“使用手册”希望为大家全面的介绍 {\LaTeX} 环境的搭建方法、{\BIThesis} 的使用方法,从而快速掌握使用 {\LaTeX} 排版引擎进行基本的论文撰写的方法,完成符合学校要求的学位论文。{\BIThesis} 目前使用 GitHub 进行维护,官方项目地址位于:

\begin{center}
\color{ForestGreen}\href{https://github.com/spencerwooo/BIThesis}{\texttt{https://github.com/spencerwooo/BIThesis}}
\end{center}

\subsection{{\BIThesis} 在线说明文档}
和本手册的目标类似,{\BIThesis} 项目同样维护了一个在线版本的说明文档,位于:{\href{https://github.com/spencerwooo/BIThesis/wiki}{BIThesis - wiki}},二者的目的、内容、功能类似,且会随着模板的开发与维护同步更新。

{\BIThesis} 在线说明文档目前拥有如下模块:

\begin{enumerate}
\item \href{https://github.com/spencerwooo/BIThesis/wiki}{主页:Home}
\item \href{https://github.com/spencerwooo/BIThesis/wiki/First-things-first}{如何开始:First things first }
\item \href{https://github.com/spencerwooo/BIThesis/wiki/Using-one-of-the-templates}{使用其中一个模板:Using one of the templates}
\item \href{https://github.com/spencerwooo/BIThesis/wiki/Proposal-Report}{本科生开题报告:Proposal report}
\item \href{https://github.com/spencerwooo/BIThesis/wiki/Final-Graduation-Thesis}{本科生毕业论文:Graduation thesis}
\item \href{https://github.com/spencerwooo/BIThesis/wiki/Lab-Report}{本科生实验报告:Lab report}
\item \href{https://github.com/spencerwooo/BIThesis/wiki/Converting-to-Word}{将 LaTeX 文档转换为 Word:Converting to Word}
\end{enumerate}

接下来,我们正式开始介绍 {\LaTeX} 与 {\BIThesis} 的使用方法。

\subsection{准备工作}

\subsection{下载合适的 \LaTeX 发行版}
\begin{minted}[
  frame=single
]{bash}
sudo apt install texlive
\end{minted}

\subsection{挑选合适的 \LaTeX 编辑器}
