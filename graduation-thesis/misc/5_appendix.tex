%%
% The BIThesis Template for Bachelor Graduation Thesis
%
% 北京理工大学毕业设计(论文)附录 —— 使用 XeLaTeX 编译
%
% Copyright 2020 Spencer Woo
%
% This work may be distributed and/or modified under the
% conditions of the LaTeX Project Public License, either version 1.3
% of this license or (at your option) any later version.
% The latest version of this license is in
%   http://www.latex-project.org/lppl.txt
% and version 1.3 or later is part of all distributions of LaTeX
% version 2005/12/01 or later.
%
% This work has the LPPL maintenance status `maintained'.
%
% The Current Maintainer of this work is Spencer Woo.
%
% Compile with: xelatex -> biber -> xelatex -> xelatex

\unnumchapter{附~~~~录}

% 设置附录编号格式
\ctexset{
  section/number = 附录\Alph{section}
}

附录相关内容…

\section{\LaTeX 环境的安装}

\LaTeX 环境的安装。

\section{BIThesis使用说明}

BIThesis使用说明。

\textcolor{blue}{附录是毕业设计(论文)主体的补充项目,为了体现整篇文章的完整性,写入正文又可能有损于论文的条理性、逻辑性和精炼性,这些材料可以写入附录段,但对于每一篇文章并不是必须的。附录依次用大写正体英文字母 A、B、C……编序号,如附录 A、附录 B。阅后删除此段。}

\textcolor{blue}{附录正文样式与文章正文相同:宋体、小四;行距:22 磅;间距段前段后均为 0 行。阅后删除此段。}

% 恢复默认编号格式
\ctexset{
  section/number = \thesection
}
