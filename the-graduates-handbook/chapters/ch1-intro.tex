\chapter{关于 \LaTeX 和 \BIThesis 的一些疑难解答}
\label{chap:what}

\section{为什么要用 \LaTeX{} 和 \BIThesis{}?}

学术、学位论文有严格的格式要求。为了更多同学的方便使用,校方一般提供大家更为熟悉
的 Word 模板。虽然 Word 确实是大家最常用的排版工具,但是:
\begin{center}
  \kaishu
  如果你有足够多使用 Word 的经历,一定会体验过「同一份 Word 文档,在不同地方打开
  就变得不同」这样的魔幻现实主义色彩的经历。
\end{center}

\LaTeX{} 是专用于高质量的学术论文排版的排版工具,能让同学们更专注于内容本身,更
自信地排版符合格式要求的学术、学位论文。\BIThesis{} 项目旨在提供一套开箱即用的、
符合北京理工大学硕士(博士)学位论文的\LaTeX{} 模板,以助力高质量的学术写作。通
过 \BIThesis{} 模板,学生可以轻松撰写符合学校格式要求的学位论文,可避免繁琐的论
文格式调整,从而将关注点更多地放在高质量的内容本身。

\section{为何需要这么多步骤,我该如何开始?}

首先,\LaTeX{} 并不是像 Word 一样的一个开箱即用的软件。\LaTeX{} 本质上是一门用于
排版的「语言」或「语法规则」。我们实际上,是以 \underline{纯文本文件}(以
\texttt{.tex} 结尾的文件)为基础,用这样的一套 \underline{拟定好的标记语法} 来
设定文字的格式,并 \underline{利用一些工具},将其转化为符合格式要求的 PDF 文档。

我们重新回顾一下这句话:

\begin{itemize}[noitemsep]
  \item \textbf{\underline{纯文本文件}} 意味着我们只需要创建一个以 \texttt{.tex}
  结尾的文件,即可开始论文内容的撰写;
  \item \textbf{\underline{拟定好的语法}} 则需要我们了解一些 \LaTeX{} 中常用的语
  法语言规则,用来以纯文本的形式描述内容的格式,从而让下面提到的工具可以根据格式
  需要,将文档转化为PDF。
  \item \textbf{\underline{利用一些工具}} 也就表示我们需要这些工具(程序),来将
  纯文本内容转化为符合格式的 PDF 文档:我们或是下载安装他们到本地,或是使用在线
  平台开箱即用;
\end{itemize}

因此,本手册也将以这样的逻辑,为大家分别介绍每处需要的知识 --- 我们将首先介绍如
何「安装这些工具」,并如何更舒服的创建、编写此「纯文本文件」(在自己的电脑上和使
用在线的编辑器是不一样的);而后,我们将在后续的章节,简单的讲述常用的「拟定好的
语法」--- 以让大家快速上手,使用 \BIThesis{} 撰写自己的毕业论文。

\section{在自己的电脑上编写论文}

在这里,我们将在自己的电脑上配置安装撰写 \LaTeX{} 的相关工具。首先,我们搞定
\underline{一些工具} 的安装,来更方便的撰写 \underline{纯文本文件} 并将其转化为
符合格式的 PDF 文档。

\paragraph{一些工具的安装} 在 \LaTeX{} 的世界中,我们的「一些工具」包括将
\LaTeX{} 源码按照格式转换为 PDF 文档的「编译器」,和支撑部分 \LaTeX{} 格式语法的
「宏包」。我们将他们统称为一个 \LaTeX{} 发行版 --- 也就是我们需要在自己的电脑上
安装的软件。

根据同学们使用的操作系统,可以安装相应的 \LaTeX{} 发行版:

\begin{itemize}[noitemsep]
  \item \textbf{Windows 或 Linux}:下载安装 \TeX{}Live。以 Windows 为例,访问
  \href{https://www.tug.org/texlive/windows.html}{TeX Live on Windows - Easy
  install},并下载运行 \texttt{install-tl-windows.exe};
  \item \textbf{macOS}:下载安装 Mac\TeX{},即访问
  \href{https://www.tug.org/mactex/mactex-download.html}{MacTeX - Downloading
  MacTeX 2022} 并下载安装 \texttt{MacTeX.pkg};
\end{itemize}

\paragraph{纯文本文件} 我们撰写的 \LaTeX{} 文档,确实是「无格式」的纯文本文档。
也因此,任何能够编辑纯文本的工具我们其实都可以使用。但是,专业的\LaTeX{} 编辑器
一般会提供 \LaTeX{} 源码的编辑和预览功能。虽然不是必要的,但是使用编辑器可以大大
提高\LaTeX{} 的使用效率。

对于 \TeX{}Live 或者 Mac\TeX,发行版自带了基础的编辑器(分别是 \TeX{}works 和
\TeX{}Shop),可直接使用。集成的编辑环境,比如 \TeX{}studio 也是推荐大家使用的。
另外,比如 VS Code 和 Vim 等通用代码编辑器,也可以借助插件的安装,提升 \LaTeX{}
的撰写体验。

到此,我们其实就可以直接使用本模板,在自己的电脑上进行论文的编写了。如果想再了解
有关在线编辑平台 Overleaf 的相关内容,请继续阅读~\ref{sec:online-overleaf} 节;
否则,大家可以直接跳转到~\ref{sec:using-bithesis} 节,了解模板的使用方法。

\section{本地编辑或是 Overleaf 在线平台,我改使用哪一个?}
\label{sec:online-overleaf}

Overleaf 是一个在线的 \LaTeX{} 编辑器,可以直接在网页上进行 \LaTeX{} 的编辑和预
览。大家可以访问注册 \url{https://overleaf.com} 使用 Overleaf 在线编写
\LaTeX{}。选用 Overleaf 有优点也有缺点:
\begin{itemize}[noitemsep]
  \item 优点在于:
    \begin{itemize}[noitemsep]
      \item 不需要安装 \LaTeX{} 发行版,不需要配置编辑器,直接在网页上进行
      \LaTeX{} 的编辑和预览。
      \item 数据保存在云端,可以在多个设备上进行编辑和预览。
      \item 可以共享项目,方便多人协作。(对于毕业论文来说,这个优点并不是很重
      要。)
    \end{itemize}
  \item 缺点在于:
    \begin{itemize}[noitemsep]
      \item 由于 Overleaf 是一个在线的编辑器,所以需要保持网络连接,否则无法进行
      编辑和预览。
      \item 很多同学使用了第三方的文献管理软件,如 Zotero。Overleaf 无法直接与这
      些软件进行集成,需要手动导入文献。
      \item 网页版的编辑器功能有限,无法进行复杂的自定义配置。
    \end{itemize}
\end{itemize}

因此,需要使用者根据自己的需求进行选择。

\section{如何将自己电脑上的论文转到 Overleaf}

\ref{sec:overleaf-compile} 节介绍了如何从 \url{https://bithesis.bitnp.net} 新建项目,那样通常更简单;不过若您已在本地用了模板,想转到 Overleaf(例如为了调试),请参考此节。

\begin{enumerate}
  \item 按网页提示上传文件到 Overleaf,注意\textbf{避免嵌套文件夹}。

    (原因:嵌套文件夹可能导致无法统计字数;不过不影响编译。)

    参考图\ref{fig:overleaf-recompile},文件 \texttt{main.tex}、文件夹 \texttt{chapters/} 等在根目录,而没有嵌套在 \texttt{graduate-thesis/} 文件夹中。若您已嵌套,可到左侧文件列表单击再拖动来移动文件。

    详细操作如下。访问 \url{https://www.overleaf.com/project},单击左上角\texttt{New Project},然后有下面两种方法。

    \begin{itemize}
      \item 选择\texttt{Blank Project},稍等片刻。待创建完成后,选择左上角\texttt{Upload}按钮,逐一上传文件(\texttt{Select files})或一次性上传文件夹(\texttt{Select a folder})。
      \item 将自己电脑中的文件夹打包成ZIP,通过\texttt{Upload Project}上传ZIP文件。
    \end{itemize}

  \item 单击左上角 \texttt{Menu} 打开侧边栏,找到 \texttt{Settings} 一段,\textbf{将 \texttt{Compiler} 一项的值改为 \texttt{XeLaTeX}}。

    (原因:默认的 \texttt{pdfLaTeX} 几乎不支持汉字,不修改则无法正常编译。)
\end{enumerate}
