%%
% The BIThesis Template for Graduate Thesis
%
% Copyright 2020-2023 Yang Yating, BITNP
%
% This work may be distributed and/or modified under the
% conditions of the LaTeX Project Public License, either version 1.3
% of this license or (at your option) any later version.
% The latest version of this license is in
%   http://www.latex-project.org/lppl.txt
% and version 1.3 or later is part of all distributions of LaTeX
% version 2005/12/01 or later.
%
% This work has the LPPL maintenance status `maintained'.
%
% The Current Maintainer of this work is Feng Kaiyu.

\begin{appendices}
  \chapter{费马大定理的证明}
  关于此,我确信已发现了一种美妙的证法,可惜这里空白的地方太小,写不下。

  \chapter{Maxwell Equations}
  因为在柱坐标系下,$\overline{\overline\mu}$是对角的,所以Maxwell方程组中电场$\bf
  E$的旋度

  所以$\bf H$的各个分量可以写为:
  \begin{subequations}
    \begin{eqnarray}
      H_r=\frac{1}{\mathbf{i}\omega\mu_r}\frac{1}{r}\frac{\partial
        E_z}{\partial\theta } \\
      H_\theta=-\frac{1}{\mathbf{i}\omega\mu_\theta}\frac{\partial E_z}{\partial r}
    \end{eqnarray}
  \end{subequations}

  同样地,在柱坐标系下,$\overline{\overline\epsilon}$是对角的,所以Maxwell方程组中磁场$\bf
  H$的旋度
  \begin{subequations}
    \begin{eqnarray}
      &&\nabla\times{\bf H}=-\mathbf{i}\omega{\bf D}\\
      &&\left[\frac{1}{r}\frac{\partial}{\partial
          r}(rH_\theta)-\frac{1}{r}\frac{\partial
          H_r}{\partial\theta}\right]{\hat{\bf
          z}}=-\mathbf{i}\omega{\overline{\overline\epsilon}}{\bf
        E}=-\mathbf{i}\omega\epsilon_zE_z{\hat{\bf z}} \\
      &&\frac{1}{r}\frac{\partial}{\partial
        r}(rH_\theta)-\frac{1}{r}\frac{\partial
        H_r}{\partial\theta}=-\mathbf{i}\omega\epsilon_zE_z
    \end{eqnarray}
  \end{subequations}

  由此我们可以得到关于$E_z$的波函数方程:
  \begin{eqnarray}
    \frac{1}{\mu_\theta\epsilon_z}\frac{1}{r}\frac{\partial}{\partial r}
    \left(r\frac{\partial E_z}{\partial r}\right)+
    \frac{1}{\mu_r\epsilon_z}\frac{1}{r^2}\frac{\partial^2E_z}{\partial\theta^2}
    +\omega^2 E_z=0
  \end{eqnarray}

  \chapter{要求}

  \textcolor{blue}{
  有些材料编入文章主体会有损于编排的条理性和逻辑性,或有碍于文章结构的紧凑和突出主题思想等,这些材料可作为附录另页排在参考文献之后,也可以单编成册。下列内容可作为附录:
  }
  \begin{enumerate}
    \item \textcolor{blue}{为了整篇论文材料的完整,但编入正文有损于编排的条理性和逻辑性的材料,这一类材料包括比正文更为详尽的信息、研究方法和技术等更深入的叙述,以及建议可阅读的参考文献题录和对了解正文内容有用的补充信息等;}
    \item \textcolor{blue}{ 由于篇幅过大或取材的复制资料不便于编入正文的材料; }
    \item \textcolor{blue}{ 不便于编入正文的罕见珍贵资料; }
    \item \textcolor{blue}{ 一般读者无须阅读,但对本专业同行有参考价值的资料; }
    \item \textcolor{blue}{ 某些重要的原始数据、推导、计算程序、框图、结构图、注释、统计表、计算机打印输出件等; }
  \end{enumerate}

  \section{一级标题}
  \subsection{二级标题}
\end{appendices}
